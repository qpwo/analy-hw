\documentclass{article}

\usepackage{graphicx} % for loading images
\usepackage{amssymb} % for naturals, reals, etc
\usepackage{enumerate} % for non-bullet lists
\usepackage{amsmath} % for aligning equations
\usepackage[in]{fullpage} % small margins
\setlength{\parindent}{0pt} % don't indent paragraphs
\pagenumbering{gobble} % don't number pages
\linespread{1.5} % bigger line spacing

\newcommand{\sP}{\mathcal P}

\title{Analysis HW 9}
\author{Luke Miles}

\begin{document}
\raggedright % right-aligned instead of justified
\renewcommand{\thefootnote}{\fnsymbol{footnote}} % cool looking footnotes

\makeatletter
\textsc{\@title\ - \@author\ - \@date}

Last homework!!! \hfill \includegraphics{Cantor-floating}
\makeatother

\hrulefill

\textbf{Exercise 7.3 - 21}:
Let $f, g \in \mathcal R[a, b]$.
\begin{enumerate}[(a)]
  \item If $t \in \mathbb R$, show that $\int_a^b (t f \pm g)^2 \geq 0$.
  \item Use (a) to show that $2 \left| \int_a^b f g \right|
    \leq t \int_a^b f^2 + (1/t) \int_a^b g^2$
  \item If $\int_a^b f^2 = 0$, show that $\int_a^b f g = 0$.
  \item Now prove that $\left| \int_a^b f g \right|^2
    \leq \left( \int_a^b |f g| \right)^2
    \leq \left( \int_a^b f^2 \right) \cdot \left( \int_a^b g^2 \right)$.
    This is called the Cauchy-Bunyakovsky-Milkshake-Schwarz Inequality.
\end{enumerate}

\underline{\textit{Solution}}:
For brevity, we omit the limits on the integrals.
\begin{enumerate}[(a)]
  \item Clearly $(t f(x) \pm g(x))^2 \geq 0$ for all $x \in [a, b]$.
    Integrating both sides gives the result.
  \item $%\displaystyle
    0
    \leq \int (t f \pm g)^2
    = \int (t^2 f^2 \pm 2tfg + g^2)
    = t^2 (\int f^2) \pm  2t (\int f g) + (\int g^2)
    \rightarrow
    \pm 2 (\int f g)
    \leq t (\int f^2) + \frac{1}{t} (\int g^2)
    $
    The conclusion follows.
  \item Since $f^2 \geq 0$, $\int f^2 = 0$ means that $f(x) = 0$ for all
    $x$. The conclusion follows.
  \item \textit{(With help from book hint.)} If $\int f^2 = 0$ then
    everything is zero. Otherwise, letting
    $t = \sqrt{\int g^2 / \int f^2}$ in (b) yields
    $2 | \int f g | \leq \sqrt{\int g^2 / \int f^2} \int f^2
    + \sqrt{\int f^2 / \int g^2} \int g^2 = 2 \sqrt{\int g^2 \int f^2}$.
    Dividing both sides by 2 and squaring shows the first is $\leq$
    the third. The second is sandwhiched in their for semi-obvious reasons.
\end{enumerate}

\hrulefill

\textbf{Exercise 7.4 - 5}:
Let $f, g, h$ be bounded functions on $I := [a, b]$ such that
$f(x) \leq g(x) \leq h(x)$ for all $x \in I$. Show that if $f$ and $h$ are
Darboux integrable and if $\int_a^b f = \int_a^b h$, then $g$ is also
Darboux integrable with $\int_a^b g = \int_a^b f$.

\underline{\textit{Solution}}:
Let $\sP$ be any partition of $[a, b]$. Then clearly
$L(f, \sP) \leq U(g, \sP)$ and $L(g, \sP) \leq U(h, \sP)$. Hence,
$L(f) \leq U(g)$ and $L(g) \leq U(h)$. And since
$L(f) = U(h)$\footnote{Because $\int_a^b g = \int_a^b f$}, we have
$U(g) = L(g)$ and hence $g$ is Darboux integrable.

\hrulefill

\textbf{Exercise 7.4 - 6}:
Let $f$ be defined on $[0, 2]$ by $f(x) := 1$ if $x \not= 1$ and
$f(1) := 0$. Show that the Darboux integral exists and find its value.

\underline{\textit{Solution}}:
Let $\varepsilon > 0$ and define the interval
$\sP = (0, 1 - \varepsilon, 1)$. Then
$L(f, \sP) = 1 \cdot ((1 - \varepsilon) - 0)
+ 0 \cdot (1 - (1 - \varepsilon)) = 1 - \varepsilon$ and
$U(f, \sP) = 1 \cdot ((1 - \varepsilon) - 0)
+ 1 \cdot (1 - (1 - \varepsilon)) = 1$. Clearly $L$ and $U$ can get
arbitrarily close to eachother (centering in on 1) by shrinking
$\varepsilon$, so $f$ is Darboux integrable and its value is 1.

\hrulefill

\textbf{Exercise 7.4 - 7}:
\begin{enumerate}[(a)]
  \item Prove that if $g(x) := 0$ for $0 \leq x \leq 1/2$ and
    $g(x) := 1$ for $1/2 < x \leq 1$, then the Darboux integral of
    $g$ on $[0, 1]$ is equal to $1/2$.
  \item Does the conclusion hold if we change the value of $g$ at the point
    $1/2$ to 13?
\end{enumerate}

\underline{\textit{Solution}}:
\begin{enumerate}[(a)]
  \item Let $\varepsilon > 0$ and define
    $\sP := (0, 1/2 - \varepsilon, 1/2 + \varepsilon, 1)$. Then
    $U(g, \sP) = 1/2 + \varepsilon$ and $L(g, \sP) = 1/2 - \varepsilon$.
    Then they get arbitrarily close and $L(g) = 1/2 = U(g)$.
  \item Yes. There is the difference is that
    $U(g, \sP) = 1/2 + 25\varepsilon$, but you can choose an even smaller
    $\varepsilon$.
\end{enumerate}

\end{document}
