\documentclass{article}

\usepackage{graphicx} % for loading images
\usepackage{amssymb} % for naturals, reals, etc
\usepackage{enumerate} % for non-bullet lists
\usepackage{amsmath} % for aligning equations
\usepackage[in]{fullpage} % small margins
\setlength{\parindent}{0pt} % don't indent paragraphs
\pagenumbering{gobble} % don't number pages
\linespread{1.5} % bigger line spacing

\title{Analysis HW 7}
\author{Luke Miles}

\begin{document}
\raggedright % right-aligned instead of justified
\renewcommand{\thefootnote}{\fnsymbol{footnote}} % cool looking footnotes

\makeatletter
\textbf{\@title\ - \@author\ - \@date}
\makeatother

\centerline{
  \includegraphics{Fibonacci}
  \includegraphics{Fibonacci}
  \includegraphics{Fibonacci}
  \includegraphics{Fibonacci}
  \includegraphics{Fibonacci}
  \includegraphics{Fibonacci}
  \includegraphics{Fibonacci}
  \includegraphics{Fibonacci}
  \includegraphics{Fibonacci}
  \includegraphics{Fibonacci}
}

\hrulefill

\textbf{Exercise 5.4 - 2}: Show that the function $f(x):=1/x^2$ is uniformly
continuous on $A:=[1,\infty)$, but that it is not uniformly continuous on
$B:= (0,\infty)$.

\underline{\textit{Solution}}: First we will show that $f$ is uniformly
continuous on $A$.  Let $\varepsilon>0$ and choose $\delta:=2\varepsilon$.
Then if $a,b\in A$ and $|a-b|<\delta$ and WLOG $a<b$, we have
\[
  |f(a)-f(b)|=\frac1{a^2}-\frac1{b^2}=\frac{b^2-a^2}{a^2b^2}
  =\frac{b+a}{a^2b^2}(b-a)<2(b-a)<2\delta=\varepsilon.
\]
Now we will show that $f$ is not uniformly continuous on $B$. Choose
$\varepsilon=1$ and let $\delta>0$ and further assume that $\delta<1/3$.
Choose $a:=\delta,b:=\frac32\delta$, and we get
\[
  |f(a)-f(b)|=|f(\delta)-f(\frac32\delta)|
  =\frac1{\delta^2}-\frac1{\frac94\delta^2}
  =\frac59\times\frac1{\delta^2}>5>\varepsilon.
\]

\hrulefill

\textbf{Exercise 5.4 - 7}: If $f(x):=x$ and $g(x):=\sin x$, show that both
$f$ and $g$ are uniformly continuous on $\mathbb R$, but that their product
$fg$ is not uniformly continuous on $\mathbb R$.

\underline{\textit{Solution}}:
\begin{itemize}
  \item First, $f$ is uniformly continuous:
    Let $\varepsilon>0$ and choose $\delta:=\varepsilon$. Then
    \[|f(x)-f(u)|=|x-u|<\delta=\varepsilon.\]
  \item To see that $g$ is uniformly continuous, notice that
    $|\sin a-\sin b|<|a-b|$ for all $a,b\in\mathbb R$. This holds because 2
    triangles drawn in the unit circle always have a greater difference in
    arc length than in height.
  \item Finally, $fg$ is not uniformly continuous. Choose $\varepsilon:=1$
    and let $\delta>0$. Now choose the smallest integer $n$ where
    $n>|1/\sin(\delta/2)|$. Choose $u:=2n\pi$ and $x:=u+\delta/2$. Then
    \[
      |fg(x)-fg(u)|
      =|(2n\pi+\delta/2)\sin(2n\pi+\delta/2)-(2n\pi)\sin(2n\pi)|
    \]\[
      =|(2n\pi+\delta/2)\sin(\delta/2)|
      >|2n\pi\sin(\delta/2)|
      >\left|\frac2{\sin(\delta/2)}\pi\sin(\delta/2)\right|
      =2\pi
      >\varepsilon
    \]
\end{itemize}

\hrulefill

\textbf{Exercise 5.4 - 14}: A function $f: \mathbb R\to\mathbb R$ is said
to be periodic on $\mathbb R$ if there exists a number $p>0$ such that
$f(x+p)=f(x)$ for all $x\in\mathbb R$. Prove that a continuous periodic
function on $\mathbb R$ is bounded and uniformly continuous on $\mathbb R$.

\underline{\textit{Solution}}: Let $f$ be a continuous periodic function on
$\mathbb R$ with period $p$ and let $I:=[0,p]$. Then
\begin{itemize}
  \item the set $f(I)$ must be a closed interval. Now let $x\in\mathbb R$.
    Then $f(x)=f(a)$ for some $a\in I$, and hence $f$ is bounded everywhere.
  \item $f$ is uniformly continuous on $I$. By a similar argument, $f$ is
    uniformly continuous everywhere.
\end{itemize}

\hrulefill

\textbf{Exercise 5.6 - 8}: Let $f,g$ be strictly increasing on an interval
$I\subseteq\mathbb R$ and let $f(x)>g(x)$ for all $x\in I$. If
$y\in f(I)\cap g(I)$, show that $f^{-1}(y)<g^{-1}(y)$.

\underline{\textit{Solution}}: Define $x_f:=f^{-1}(y),x_g:=g^{-1}(y)$.
Suppose that $x_f\geq x_g$.  Then, because $g$ is strictly increasing,
$g(x_f)\geq g(x_g)=y=f(x_f)$. Now we have the clear contradiction
$g(x_f)\geq f(x_f)$.
\hfill $\blacksquare$

\hrulefill

\textbf{Exercise 5.6 - 10}: Let $I:=[a,b]$ and let $f:I\to\mathbb R$ be
continuous on $I$. If $f$ has an absolute maximum [respectively, minimum]
at an interior point $c$ of $I$, show that $f$ is not injective on $I$.

\underline{\textit{Solution}}: WLOG, assume $c$ is an absolute maximum and
$b>a$. Choose a small enough $\delta$ so that $f$ is increasing over
$I_1:=(c-\delta,c)$ and decreasing over $I_2:=(c,c+\delta)$. Then
$S:=f(I_1)\cap f(I_2)$ is either empty or nonempty. If it is nonempty, then
there exists $a\in I_1,b\in I_2$ so that $f(a)=f(b)$, and hence $f$ is not
injective. If $S$ is empty, then one of $I_1$ and $I_2$ are constant under
$f$, and again $f$ is not injective.
\hfill $\blacksquare$

\hrulefill

\textbf{Exercise 5.6 - 12}: Let $f:[0,1]\to\mathbb R$ be a continuous
injective function with $f(0)<f(1)$. Show that $f$ is strictly increasing
on $[0,1]$.

\underline{\textit{Solution}}: Let $a,b\in [0,1]$ with $a<b$. If $f(a)<f(b)$
we are done, $f(a)=f(b)$ is impossible because $f$ is injective, and so we
consider $f(a)>f(b)$. Define $I:=[a,b]$ and consider $m:=\max f(I)$.
Either $m$ is inside $I$, or $m=a$. If $m$ is internal then exercise 5.6 -
10 shows that $f$ is not injective and we have a contradiction. If $m=a$
then slide $a$ backwards until $\max I$ is not an endpoint.

%todo: improve this argument

\hrulefill

\textbf{Exercise 6.1 - 1}:

Use the definition to find the derivative of each of the following
functions:
\begin{enumerate}[(a)]
  \item $f(x):=x^3$ for $x\in\mathbb R$? Let $c\in\mathbb R$ and define
    $L:=3c^2$. Let $\varepsilon>0$ and choose $\delta=\varepsilon/(4c)$.
    Then
    \[
      \left|\frac{f(x)-f(c)}{x-c}-L\right|
      =\left|\frac{x^3-c^3}{x-c}-3c^2\right|
      =|x^2+cx+c^2-3c^2|
      =|x^2+cx-2c^2|
    \]\[
      <|(c+\delta)^2+c(c+\delta)-2c^2|
      =|\delta^2+3c\delta|
      <|4c\delta|
      =|4c\frac{\varepsilon}{4c}|
      =\varepsilon
    \]
  \item $g(x):=1/x$ for $x\in\mathbb R,x\not=0$? Let $c\in\mathbb R$ and
    define $L:=-1/c^2$. Let $\varepsilon>0$ and choose
    $\delta=(c^3\varepsilon)/(1+c^2\varepsilon)$. Then
    \[
      \left|\frac{g(x)-g(c)}{x-c}-L\right|
      =\left|\frac{1/x-1/c}{x-c}+1/c^2\right|
      =\left|\frac{x-c}{c^2x}\right|
      <\left|\frac{(c+\delta)-c}{c^2x}\right|
    \]\[
      =\left|\frac{\delta}{c^2x}\right|
      <\footnote{Assuming $c>0$. Just switch to $c+\delta$ for $c<0$.}
      \left|\frac{\delta}{c^2(c-\delta)}\right|
      =\left|\frac{(c^3\varepsilon)/(1+c^2\varepsilon)}
                  {c^2(c-(c^3\varepsilon)/(1+c^2\varepsilon))}\right|
      =\varepsilon
    \]
  %\item $h(x):=\sqrt x$ for $x>0$? Let $c\in(0,\infty)$ and define
  %  $L:=1/(2\sqrt c)$. Let $\varepsilon>0$ and choose $\delta=\dots$. Then
  %  \[
  %    \left|\frac{h(x)-h(c)}{x-c}-L\right|
  %    =\left|\frac{\sqrt x-\sqrt c}{x-c}-\frac1{2\sqrt c}\right|
  %    =\left|\frac{\sqrt c-\sqrt x}{2(c+\sqrt{cx})}\right|
  %    <\left|\frac{\sqrt c-\sqrt x}{c+\sqrt{cx}}\right|
  %    <\left|\frac{c-x}{c+\sqrt{cx}}\right|
  %  \]\[
  %    <\left|\frac{c-(c+\delta)}{c+\sqrt{cx}}\right|
  %    =\frac{\delta}{c+\sqrt{cx}}
  %    <\frac{\delta}{c+\sqrt{c^2+2cx+x^2}}
  %    =\frac{\delta}{2c+x}
  %    <\frac{\delta}{3c-\delta}
  %    <\frac{\delta}{c-\delta}
  %    =\frac{(c\varepsilon)/(1+\varepsilon)}
  %          {c-(c\varepsilon)/(1+\varepsilon)}
  %    =\varepsilon
  %  \]
  %\item $k(x):=1/\sqrt x$ for $x>0$? Let $c\in(0,\infty)$ and define
  %  $L:=-1/(2c\sqrt c)$. Let $\varepsilon>0$ and choose $\delta=\dots$. Then
  %  \[
  %    \left|\frac{k(x)-k(c)}{x-c}-L\right|
  %    =\left|\frac{1/\sqrt x-1/\sqrt c}{x-c}+\frac1{2c\sqrt c}\right|
  %    =\dots
  %  \] %todo
\end{enumerate}
\textit{Parts c and d were proven unsolvable by Gauss.}

\hrulefill

\textbf{Exercise 6.1 - 4}: Let $f:\mathbb R\to\mathbb R$ be defined by
$f(x):=x^2$ for $x$ rational, $f(x):=0$ for $x$ irrational. Show that $f$ is
differentiable at $x=0$, and find $f'(0)$.

\underline{\textit{Solution}}: \textit{With help from hint in back of book.}
Two in one! The function is differentiable because the following limit
exists.
\[
  \lim_{x\to c}\frac{f(x)-f(c)}{x-c}
  =\lim_{x\to0}\frac{f(x)-f(0)}{x-0}
  =\lim_{x\to0}\frac{f(x)}{x}
  =\footnote{Because $0\leq|f(x)/x|\leq|x|$ for all $x\in\mathbb R$}
  \lim_{x\to0}x
  =0
\]

\hrulefill

\textbf{Exercise 6.1 - 9}: Prove that if $f:\mathbb R\to\mathbb R$ is an
even function [that is, $f(-x)=f(x)$ for all $x\in\mathbb R$] and has a
derivative at every point, then the derivative $f'$ is an odd function
[that is, $f'(-x)=-f'(x)$ for all $x\in\mathbb{R}$]. Also prove that if
$g:\mathbb R\to\mathbb R$ is a differentiable odd function, then $g'$ is an
even function.

\underline{\textit{Solution}}: Let $c\in\mathbb{R}$.
\[
  f'(-c)
  =\lim_{x\to-c}\frac{f(x)-f(-c)}{x-(-c)}
  =\lim_{x\to-c}\frac{f(x)-f(c)}{x+c}
  =\lim_{x\to c}\frac{f(-x)-f(c)}{-x+c}
  =\lim_{x\to c}\frac{f(x)-f(c)}{-x+c}
  =-f'(c)
\]
\[
  g'(-c)
  =\lim_{x\to-c}\frac{f(x)-f(-c)}{x-(-c)}
  =\lim_{x\to-c}\frac{f(x)+f(c)}{x+c}
  =\lim_{x\to c}\frac{f(-x)+f(c)}{(-x)+c}
  =\lim_{x\to c}\frac{-f(x)+f(c)}{-x+c}
  =g'(c)
\]

\hrulefill

\textbf{Exercise 6.1 - 15}: Given that the restriction of the cosine
function $\cos$ to $I:=[0,\pi]$ is strictly decreasing and that
$\cos0=1,\cos\pi=-1$, let $J:=[-1,1]$, and let $\arccos: J\to\mathbb R$ be
the function inverse to the restriction of $\cos$ to $I$. Show that the
$\arccos$ is differentiable on $(-1,1)$ and $D \arccos y=-1/\sqrt{1-y^2}$
for $y\in(-1,1)$. Show that $\arccos$ is not differentiable at $-1$ and $1$.

\underline{\textit{Solution}}: Suppose that $x=\arccos y$. Then $\cos x=y$.
Taking the derivative of both sides with respect to $y$ yields
$-\sin y\times\frac{dx}{dy}=1$. Dividing through by $-\sin y$, we have our
desired result:
\[
  \frac{d\arccos y}{dy}
  =\frac{dx}{dy}=\frac{-1}{\sin y}
  =\frac{-1}{\sqrt{1-\cos^2x}}
  =\frac{-1}{\sqrt{1-y^2}}
\]
Clearly this is well defined for all $x\in(0,1)$. The derivative does not
exist at $x=0$ or $x=1$ because $\arccos$ is not continuous there (0 and
1 are endpoints).

\hrulefill

\textbf{Exercise 6.2 - 2}: Find the points of relative extrema, the
intervals on which the following functions are increasing, and those on
which they are decreasing. \textit{Since the problem asks to
\underline{find} the values, I provide minimal explanation.}
\begin{enumerate}[(a)]
  \item $f(x):=x+1/x$ for $x\not=0$? $f$ has a relative maximum of -2 at
    $x=-1$ and a relative minimum of 2 at $x=1$, both holding inside of 
    $\delta=1/2$.  $f$ is increasing over $(\infty,-1)$ and $(1,\infty)$
    and decreasing over $(-1,0)$ and $(0,1)$. Changes occur at -1 and 1
    because $|1/x|>|x|$ only if $|x|<1$.
  \item $g(x):=x/(x^2+1)$ for $x\in\mathbb{R}$? $g$ has a relative minimum
    of -1/2 at $x=-1$ and a relative maximum of 1/2 at $x=1$, both holding
    inside of $\delta=1/2$. $g$ is increasing on $(-1,1)$ and decreasing on
    $(-\infty,-1)$ and $(1,\infty)$. Similar to $f$, -1 and 1 are critical
    points because $|x|<x^2$ only if $|x|<1$.
  \item $h(x):=\sqrt x-2\sqrt{2+x}$ for $x>0$? $h$ has a relative (and
    absolute) maximum of $-\sqrt6$ at $x=2/3$, again holding within
    $\delta=1/2$. $h$ has no relative minimums.
    $h$ is increasing over $(0,2/3)$ and decreasing over $(2/3,\infty)$.
  \item $k(x):=2x+1/x^2$ for $x\not=0$?
    $k$ has no relative maximums, but does have a relative minimum of 3 at
    $x=1$. $h$ increases over $(-\infty,0)$ and $(1,\infty)$ and decreases
    over $(0,1)$. You might expect $x=-1$ to be a critical point because
    of the $1/x^2$ term, but the curve is grabbed and pulled down by $2x$
    and the function ends up being monotone through that point.
\end{enumerate}

\hrulefill

\textbf{Exercise 6.2 - 4}: Let $a_1,a_2,\dots,a_n$ be real numbers and let 
$f$ be defined on $\mathbb R$ by
\[f(x):=\sum_{i=1}^n(a_i-x)^2\text{\ for\ }x\in\mathbb R.\]
Find the unique point of relative minimum for $f$.

\underline{\textit{Solution}}:
Since $f(x)=\sum(x-a_i)^2=nx^2-2x\sum a_i+\sum a^2_i$ is a simple
function of the form $ax^2+bx+c$, it has an absolute minimum of
$-b/(2a)=(2\sum a_i)/(2n)=(\sum a_i)/n$.
\hfill $\blacksquare$

\hrulefill

\textbf{Exercise 6.2 - 10}: Let $g:\mathbb R\to\mathbb R$ be defined by
$g(x):=x+2x^2\sin(1/x)$ for $x\not=0$ and $g(0):=0$. Show that $g'(0)=1$,
but in every neighborhood of 0 the derivative $g'(x)$ takes on both
positive and negative values. Thus $g$ is not monotonic in any neighborhood
of 0.

\underline{\textit{Solution}}: \textit{With help from book hint.}
The derivative at 0:
\[
  \lim_{x\to0}g'(x)
  =\lim_{x\to0}D(x\times(1+2x\sin\frac1x))
  =\lim_{x\to0}xD(1+2x\sin\frac1x)+1+2x\sin\frac1x
  =\lim_{x\to0}1+2x\sin\frac1x
  =1+0
  =1
\]
The derivative elsewhere:
\[
  g'(x)
  =D(x+2x^2\sin\frac1x)
  =1+D(2x^2\sin\frac1x)
  =1+4x\sin\frac1x+2x^2D(\sin\frac1x)
  =1+4x\sin\frac1x-2\cos\frac1x
\]
Let $\delta>0$ and assume $\delta<1/10$. Then choose an $n$ such
that $x:=1/(n\pi)<\delta$. Then clearly, depending on whether $n$ is odd
or even, $g'(x)$ can be positive or negative.

\hrulefill

\textbf{Exercise 6.2 - 15}: Let $I$ be an interval. Prove that if $f$ is
differentiable on $I$ and if the derivative $f'$ is bounded on $I$, then
$f$ satisfies a Lipschitz condition on $I$.

\underline{\textit{Solution}}: Let $x,c\in I$. Then, because the derivative
is bounded, there exists a natural number $K$ so that
$|\frac{f(x)-f(c)}{x-c}|<K$. A little algebra proves the result:
\[
  \left|\frac{f(x)-f(x)}{x-c}\right|
  =\frac{|f(x)-f(c)|}{|x-c|}
  <K
  \Rightarrow |f(x)-f(c)|<K|x-c|
\]

\hrulefill

\textbf{Exercise 6.3 - 8}: Evaluate the following limits:
\begin{enumerate}[(a)]
  \item $\lim_{x\to0}\frac{\arctan x}x\ (-\infty,\infty)$?
    Applying L'Hospital's rule, we get
    $\lim_{x\to0}\frac1{1+x^2}=1$.
  \item $\lim_{x\to0}\frac1{x(\ln x)^2}\ (0,1)$? We can rewrite it as
    $\frac{1/x}{(\ln x)^2}$ and apply L'Hospital to get
    $\frac{-1/x^2}{2\ln x/x}=\frac{1}{2}\frac{1/x}{-\ln x}$. Applying
  L'Hospital again, we have $\frac12\frac{-1/x^2}{-1/x}=\frac1{2x}$.
  Finally, we get $\lim_{x\to0+}\frac1{2x}=\infty$.
  \item $\lim_{x\to0+}x^3\ln x\ (0,\infty)$?
    Rewrite as $\frac{\ln x}{1/x^3}$ and apply L'Hospital's rule to get
    $\frac{1/x}{-3/x^4}=\frac{-x^3}{3}$ and we have
    $\lim_{x\to0+}\frac{-x^3}{3}=0$.
  \item $\lim_{x\to\infty}\frac{x^3}{e^3}\ (0,\infty)$? $\infty$.
\end{enumerate}

\hrulefill

\textbf{Exercise 6.4 - 3}: Use induction to prove Leibniz's rule for the
$n$th derivative of a product:
\[(fg)^{(n)}(x)=\sum_{k=0}^n\binom nkf^{(n-k)}(x)g^{(k)}(x).\]

\underline{\textit{Solution}}: Equality clearly holds for $n=1$. Now suppose
that the equation is true for all $n\leq j$. We will show it also holds for
$n=j+1$. For brevity, we omit the ``of x'' (x) and express differentiation
with normal looking exponents.
\begin{flalign*}
  && (fg)^{j+1}&=\left((fg)^j\right)^{1} &\\
  && &=\frac d{dx}\sum_{k=0}^j\binom jkf^{j-k}g^{k} &\\
  && &=\sum_{k=0}^j\frac d{dx}\binom jkf^{j-k}g^{k} &\\
  && &=\sum_{k=0}^j\binom jk\left(f^{j-k+1}g^{k}+f^{j-k}g^{k+1}\right) &\\
  && &=\sum_{k=0}^{j+1}\binom{j+1}kf^{j-k+1}g^{k} &\blacksquare\\
\end{flalign*}

\hrulefill

\textbf{Exercise 6.4 - 10}: Let $h(x):=e^{-1/x^2}$ for $x\not=0$ and
$h(0):=0$. Show that $h^{(n)}(0)=0$ for all $n\in\mathbb N$. Conclude that
the remainder term in Taylor's Theorem for $x_0=0$ does \textit{not}
converge to zero as $n\to\infty$ for $x\not=0$. (hint in book)

\underline{\textit{Solution}}:
\begin{itemize}
  \item Note the following:
    \[
      \lim_{x\to0}\frac{h(x)}{x^k}
      =\lim_{x\to0}\frac{e^{-1/x^2}}{x^k}
      =\footnote{L'Hospital's rule applies.}
      \lim_{x\to0}\frac{\frac2{x^3}e^{-1/x^2}}{kx^{k-1}}
      =\frac2k\lim_{x\to0}\frac{e^{-1/x^2}}{x^{k+2}}
      \Rightarrow\lim_{x\to0}\frac{e^{-1/x^2}}{x^k}=0
    \]
    And since every $h^{(n)}(x)$ is some composition of products and
    additions of $\frac{h(x)}{x^k}$, we know $h^{(n)}(0)=0$.
  \item Let $n\in\mathbb{N}$. If we choose $x_0=0$, then
    \[
      h(x)
      =h(0)+h'(0)x+\frac{h''(0)}{2!}x^2+\dots
      +\frac{h^{(n)}(0)}{n!}x^n+\frac{h^{(n+1)}(c)}{(n+1)!}x^{n+1}
      =\frac{h^{(n+1)}(c)}{(n+1)!}x^{n+1}
    \]
    In fact, the constant term is constant as $n$ increases, and hence
    clearly does not converge to 0.
\end{itemize}

\hrulefill

\textbf{Exercise 6.4 - 22}: The equation $\ln x=x-2$ has two solutions.
Approximate them using Newton's Method. What happens if $x_1:=\frac12$ is
the initial point?

\underline{\textit{Solution}}: Put the equality in the form
$\ln x-x+2=0$ and define $f(x):=\ln x-x+2$. Also define the recurrence
\[x_{n+1}:=x_n-\frac{f(x_n)}{f'(x_n}=x_n-\frac{\ln x-x+2}{1/x-1}.\]
Choosing $x_1=1/6$ gives the first four terms of
$(0.166667,0.158352,0.158594,0.158594)$. Choosing $x_1=3$ gives
$(3,3.14792,3.14619,3.14619)$. Hence, the two solutions are roughly
$x=0.158954$ and $x=3.14619$.

If $x_1=1/2$, then $x_2$ is negative and $x_3$ is complex.

\hrulefill

;)

\end{document}
