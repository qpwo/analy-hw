\documentclass{article}

\usepackage{graphicx} % for loading images
\usepackage{amssymb} % for naturals, reals, etc
\usepackage{enumerate} % for non-bullet lists
\usepackage{amsmath} % for aligning equations
\usepackage[in]{fullpage} % small margins
\setlength{\parindent}{0pt} % don't indent paragraphs
\pagenumbering{gobble} % don't number pages
\linespread{1.5} % bigger line spacing

\title{Analysis HW 6}
\author{Luke Miles}

\begin{document}
\raggedright % right-aligned instead of justified
\renewcommand{\thefootnote}{\fnsymbol{footnote}} % cool looking footnotes

\makeatletter
\textbf{\@title\ - \@author\ - \@date}
\makeatother

\centerline{
\includegraphics{Fibonacci}
\includegraphics{Fibonacci}
\includegraphics{Fibonacci}
\includegraphics{Fibonacci}
\includegraphics{Fibonacci}
\includegraphics{Fibonacci}
\includegraphics{Fibonacci}
\includegraphics{Fibonacci}
\includegraphics{Fibonacci}
\includegraphics{Fibonacci}
}

\hrulefill

\textbf{Exercise 4.2 - 4}: Prove that $\lim\limits_{x \to 0} \cos(1/x)$
does not exist but that $\lim\limits_{x \to 0} x \cos(1/x) = 0$.

\underline{\textit{Solution}}:
If $\cos(1/x)$ approaches L at 0, then for any sequence $(x_n)$ that
approaches 0 without reaching it, the sequence $(f(x_n)) := (\cos (1/x_n))$
should approach L.  Define $x_n := 1/n$. Then $(f(x_n)) = (\cos n)$, which
clearly does not converge! Hence, this limit can not exist.

Now we will show $\lim\limits_{x \to 0} x \cos(1/x) = 0$. Let
$\varepsilon > 0$ and choose $\delta := \varepsilon$. Then
\[ |f(x)| = |x \cos (1/x)| \leq |x| < \delta = \varepsilon \]

\hrulefill

\textbf{Exercise 4.2 - 12}: 
Let $f: \mathbb R \to \mathbb R$ be such that $f(x+y)=f(x)+f(y)$ for all
$x, y$ in $\mathbb R$. Assume that $\lim\limits_{x \to 0} f = L$ exists.
Prove that $L=0$, and then prove that $f$ has a limit at every point
$c \in \mathbb R$.

\textit{With assistance from Kathleen Bell}

\underline{\textit{Solution}}:

By the definition of a limit, we have
$\forall \varepsilon > 0: \exists \delta > 0: \forall x \in \mathbb R:
|x| < \delta \Rightarrow |f(x) - L| < \varepsilon$.
Then $0 < |x| < \delta/2 \rightarrow 0 < |2x| < \delta \rightarrow
|f(2x) - L| < \varepsilon \rightarrow$
\footnote{Because $f(2x) = f(x+x) = f(x)+f(x) = 2f(x)$.}
$|2f(x) - L| < \varepsilon
\rightarrow |f(x) - L/2| < \varepsilon/2 < \varepsilon$. So the function
also approaches $L/2$. Hence, $L$ must be 0.

The limit exists everywhere because $f$ is continuous:
$\lim\limits_{x \to c} |f(x) - f(c)|
= \lim\limits_{x \to c} |f(x-c)|
= \lim\limits_{x \to 0} |f(x)|
= 0$.

\hrulefill

\textbf{Exercise 4.3 - 5}: Evaluate the following limits, or show that they
do not exist.

\begin{enumerate}[(a)]
  \item $\lim\limits_{x \to 1+} \frac{x}{x-1}$: $\infty$.
    Let $\alpha > 1$. Then for
    any $x$ where $1<x<\alpha$,
    \[ f(x) = \frac{x}{x-1}
      <\footnote{for any $1<a<b$, we know $\frac 1{a-1} > \frac b{b-1}$}
      \frac{\frac{\alpha}{\alpha-1}}{\frac{\alpha}{\alpha-1}-1}
      =\alpha
    \]
    So since $f(x)$ can get arbitrarily large, we call the limit $\infty$.
  \item $\lim\limits_{x \to 1} \frac x{x-1}$: DNE. By the above argument,
    approaching 1 from below yields $+\infty$. A similar argument shows
    that approaching 1 from below yields $-\infty$, and hence the limit
    does not exist.
  \item \[ \lim_{x \to 0+} (x+2)/\sqrt{x}
      = \lim_{x \to 0+} \sqrt{x} + 2/\sqrt{x}
      > \lim_{x \to 0+} 2/\sqrt{x}
    \]
    The last term clearly tends toward $+\infty$ and so the first term must
    as well.
    % TODO: shrink vertical spacing above this problem
  \item \[
      \lim_{x\to\infty} \frac{x+2}{\sqrt x}
      = \lim_{x\to\infty} \sqrt x + \frac2{\sqrt x}
      = \left[ \lim_{x\to\infty} \sqrt x \right]
      + \left[ \lim_{x\to\infty} \frac2{\sqrt x} \right]
      = \left[ \lim_{x\to\infty} \sqrt x \right] + 0
    \]
    Again, the last term clearly tends towards $+\infty$ and so the first
    must as well.
  \item $\lim\limits_{x\to0} \sqrt{x+1}/x$: DNE. Bound $x$ so 
    $-1/2 < x < 1/2$. Then from the left, the fraction tends towards
    $-\infty$:
    \[\frac{\sqrt{x+1}}x < \frac{\sqrt x}x = \frac1{x}\]
    And likewise, from the right it tends towards $+\infty$.
    Hence, the limit does not exist.
  \item Assume $x > 1$. Then
    \[
      \lim_{x\to\infty} \frac{\sqrt{x+1}}x
      \leq \lim_{x\to\infty} \frac{\sqrt{x+x}}x
      = \lim_{x\to\infty} \frac{\sqrt{2x}}x
      = \lim_{x\to\infty} \frac{\sqrt 2}{\sqrt x}
      = 0
    \]
    And since the function is strictly positive, the limit must be 0.
  \item
    \[
      \lim_{x\to\infty} \frac{\sqrt x - 5}{\sqrt x + 3}
      = \lim_{x\to\infty} \frac{\sqrt x - 5}{\sqrt x + 3} \times
      \frac{\sqrt x-3}{\sqrt x-3}
      = \lim_{x\to\infty} \frac{x-8\sqrt x+15}{x-9}
    \]
    \[
      = \left[ \lim_{x\to\infty} \frac x{x-9} \right]
      - \left[ \lim_{x\to\infty} \frac{8\sqrt x}{x-9} \right]
      + \left[ \lim_{x\to\infty} \frac{15}{x-9} \right]
      = 1 + 0 + 0
      = 0
    \]
  \item burble
    \[
      \lim_{x\to\infty} \frac{\sqrt x - x}{\sqrt x + x}
      = \lim_{x\to\infty} \frac{\sqrt x - x}{\sqrt x + x}
      \times \frac{1/x}{1/x}
      = \lim_{x\to\infty} \frac{1/\sqrt x-1}{1/\sqrt x + 1}
      = \frac{\lim\limits_{x\to\infty} 1/\sqrt{x}-1}
             {\lim\limits_{x\to\infty} 1/\sqrt{x}+1}
      = \frac11
      = 1
    \]
\end{enumerate}

\hrulefill

\textbf{Exercise 4.3 - 13}: Let $f$ and $g$ be defined on $(a, \infty)$
and suppose $\lim\limits_{x\to\infty}f=L$ and
$\lim\limits_{x\to\infty}g=\infty$. Prove that
$\lim\limits_{x\to\infty}f\circ g = L$.

\underline{\textit{Solution}}: Expanding the definitions:
\[\forall\varepsilon>0:\exists K>0:\forall x>K: |f(x)-L|<\varepsilon\]
\[\forall\alpha>0:\exists K>0:\forall x>K: g(x)>\alpha\]
So say we need a specific value $K_0$ in order to get $f$ within
$\varepsilon$ of $L$. Then clearly $g$ can get that large because
$g$ can get larger than any $\alpha$. \hfill $\blacksquare$

\hrulefill

\textbf{Exercise 5.1 - 4}: Define $[x]$ as the floor of $x$. Determine the
points of continuity of the following functions:
\textit{Every function of the form $[f(x)]$ is constant (and hence
continuous) unless $f(x)$ ``passes by'' an integer. In each case, it is
sufficient to show when this occurs.}

\begin{enumerate}[(a)]
  \item $f(x):=[x]$? Continuous everywhere except integer $x$.
  \item $g(x):=x[x]$? Same as (a). Multiplying by a completely continuous
    function does not effect points of continuity.
  \item $h(x):=[\sin x]$? Continuous everywhere except when $\sin x$ is 0 or
    1, which is when $x$ is a number of the form $n\pi$ or $2n\pi+\pi/2$.
    Discontinuities do not occur at $\sin x = -1$ because $[-.999]=-1$.
  \item $k(x):=[1/x]$? Assuming $x \not= 0$, the inside is an integer
    whenever $x$ is of the form $1/n$. Hence, $k(x)$ is continuous
    everywhere except $x=1/n$.
\end{enumerate}

\hrulefill

\textbf{Exercise 5.1 - 12}: Suppose that a function
$f:\mathbb R\to\mathbb R$ is continuous on $\mathbb R$ and that
$f(r)=0$ for every rational number $r$. Prove that $f(x)=0$ for all
$x\in\mathbb R$

\underline{\textit{Solution}}: Let $x\in\mathbb{R}$.  If $f$ is continuous,
then for any sequence $(x_n)$ that approaches $x$, whatever $(f(x_n))$
approaches must be $f(x)$.  Consider an infinite decimal expansion
$d_1,d_2,\dots$ of $x$ and define $x_n:=d_1,d_2,\dots,d_n$. Clearly every
$x_n$ is rational so $\lim f(x_n) = 0$. And since $\lim x_n = x$, $f(x)=0$.
\hfill $\blacksquare$

\hrulefill

\textbf{Exercise 5.1 - 14}: Let $A:=(0,\infty)$ and define
$k:A\to\mathbb R$ as $k(x)=0$ for irrational $x$ and $k(a/b)=b$ for
relatively prime $a$ and $b$. Prove that k is unbounded on every open
interval in A. Conclude that k is not continuous at any point of A.

\underline{\textit{Solution}}: Let $I:=(c,d)$ be an arbitrary interval in
$A$ and let $\alpha>0$ and choose $K>\alpha$. Choose an arbitrary rational
$r$ in $I$ (we've shown in class that there is one) that also isn't an
endpoint. Consider the quantity $r\times K/(K+1)$. Either this is in the
interval and has denominator greater than $\alpha$, is too small (in
magnitude) for the interval, or has too small a denominator. In the latter
two cases, one can clearly increase $K$ until the problem goes away.
\hfill {\tiny $\blacksquare$}

\hrulefill

\textbf{Exercise 5.2 - 8}: Let $f,g$ be continuous from $\mathbb R$ to
$\mathbb R$, and suppose  that $f(r)=g(r)$ for all rational numbers $r$. 
Is it true that $f(x)=g(x)$ for all $x\in\mathbb R$?

\underline{\textit{Solution}}: Yes. Let $x$ be an irrational number and let
$(x_n)$ be a sequence that approaches $x$. Then since $f(x_n)=g(x_n)$ for
all $n$, we know that $\lim f(x_n) = \lim g(x_n)$. And because $g$ and $f$
are continuous, $f(x)=g(x)$. \hfill $\blacksquare$

\hrulefill

\textbf{Exercise 5.2 - 13}: Let $f$ be a continuous additive function on
$\mathbb R$ and let $c:=f(1)$. Show that $f(x)=cx$ for all
$x\in\mathbb R$.

\underline{\textit{Solution}}: For any natural number $n$,
$f(n)=f(1+1+\dots+1)=f(1)+f(1)+\dots+f(1)=n\times f(1)=n\times c$.
For any rational number $a/b$, we have $f(a/b)=a\times f(1/b)
=\footnote{because $f(1)=f(b/b)=b\times f(1/b) \to f(1/b)=f(1)/b$}
a/b\times f(1)=a/b\times c$. By continuity (define a sequence of rationals
that approaches some irrational) we know $f(x)=x\times f(1)$ for all
$x\in\mathbb R$ \hfill $\blacksquare$

\hrulefill

\textbf{Exercise 5.2 - 14}: Let $g:\mathbb R\to\mathbb R$ satisfy the
relation $g(x+y)=g(x)g(y)$ for all $x,y\in\mathbb R$. Show that if
$g$ is continuous at $x=0$, then $g$ is continuous at every point of
$\mathbb R$. Also if we have $g(a)=0$ for some $a\in\mathbb R$, then
$g(x)=0$ for all $x\in\mathbb R$.

\underline{\textit{Solution}}: Suppose that $g$ is continuous at 0. Then
for any sequence $(x_n)$ that approaches 0, $(g(x_n))$ approaches
$L:=g(0)$. Let $y\in\mathbb R$ and let $(y_n)$ be a sequence that
approaches $y$. Then
\[g(-y)\times g(y) = g(-y+y)=g(0)=\lim g(y_n-y)
=\lim (g(y_n)\times g(-y))=g(-y)\times\lim g(y_n)\] implies that
$\lim g(y_n)=g(y)$. Hence, $g$ is continuous everywhere in $\mathbb R$.

Now suppose that for some $a\in\mathbb R$, $g(a)=0$. Then for any
$x\in\mathbb R$,
\[g(x)=g(x+(a-a))=g(x)\times g(a-a)=g(x)\times (g(a) \times g(-a))
= g(x) \times (0 \times g(-a)) = 0\]
%TODO put a blacksquare on end of line

\hrulefill

\textbf{Exercise 5.3 - 6}: Let $f$ be continuous on the interval $[0,1]$ to
$\mathbb R$ and such that $f(0)=f(1)$. Prove that there exists a point $c$
in $[0,\frac12]$ such that $f(c)=f(c+\frac12)$. Conclude that there are,
at any time, antipodal points on the earth's equator that have the same
temperature.

\underline{\textit{Solution}}: Consider $g(x):=f(x)-f(x+\frac12)$,
defined on the interval $[0,\frac12]$.
Note that $g$ cannot always be strictly positive because that would mean
$f$ is decreasing and $f(0)>f(1)$. Likewise, $g$ cannot always be strictly
negative. Hence, $g$ must pass by zero (intermediate value theorem)
and there is a point $c\in[0,\frac12]$ where $g(c)=0$. At that same point
$c$, $f(c)=f(c+\frac12)$.

Now for the earth part, make a map between the circle and the interval [0,1]
by multiplying by the appropriate constant. Temperature varies roughly
continuously. The conclusion follows.

\hrulefill

\textbf{Exercise 5.3 - 13}: Suppose that $f:\mathbb R\to\mathbb R$ is
continuous on $\mathbb R$ and that $\lim\limits_{x\to-\infty}f=0$ and
$\lim\limits_{x\to\infty}f=0$. Prove that $f$ is bounded on $\mathbb R$
and attains either a maximum or a minimum on $\mathbb R$. Give an example to
show that both a maximum and a minimum need not be attained.

\underline{\textit{Solution}}: Because of the limits, we know that for any
$\varepsilon>0$ there exists an $M>0$ where if $|x|>M$ then
$|f(x)|<\varepsilon$. Choose $\varepsilon:=1$ and the appropriate $M$.
Then $f$ is clearly bounded when $|x|>M$. It is also bounded when
$x\in[-M,M]$ because any continuous function from a closed interval is
bounded. Hence, $f$ is bounded everywhere.

To show that $f(x)$ has an absolute maximum and an absolute minimum,
consider $f$'s value on the interval $[-M,M]$. We've shown in class that a
continous functions image from an interval has an absolute maximum and
minimum. Let's call the largest value (in magnitude) in the interval $
\hfill $\blacksquare$

Example function: $f(x)=1/(x^2+1)$.

\end{document}
