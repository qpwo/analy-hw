\documentclass{article}

\usepackage[in]{fullpage} % small margins
\usepackage{amssymb} % for naturals and reals
\usepackage{enumerate} % for non-bullet lists
\usepackage{amsmath} % for left aligned text on equations

\setlength{\parindent}{0pt}
%\pagenumbering{gobble}
\linespread{1.5}

\title{Analysis HW 5}
\author{Luke Miles}

\begin{document}
\raggedright

\makeatletter
\textbf{\@title\ - \@author\ - \@date}
\makeatother

\hrulefill

\textbf{Exercise 3.5 - 5}: If $x_n:=\sqrt{n}$, show that $(x_n)$ satisfies
$\lim |x_{n+1}-x_n| = 0$, but that it is not a Cauchy sequence.

\underline{\textit{Solution}}: First, we will show the limit. Let
$\varepsilon > 0$ and $n > 1/\varepsilon^2$ Then
\[\left| \sqrt{n+1} - \sqrt{n} \right|
  %= \sqrt{n+1} - \sqrt{n}
  = \left(\sqrt{n+1} - \sqrt{n}\right)
  \frac{\sqrt{n+1} + \sqrt{n}}{\sqrt{n+1} + \sqrt{n}}
  = \frac{n+1-n}{\sqrt{n+1} + \sqrt{n}}
  = \frac{1}{\sqrt{n+1} + \sqrt{n}}
  < \frac{1}{\sqrt{n}}
  < \frac{1}{\sqrt{1/\varepsilon^2}}
  = \varepsilon.
\]

Now, for contradiction, suppose that $(x_n)$ is Cauchy. Then for any
$\varepsilon > 0$ there is some number $H$ so that for any $n,m > H$,
we have $|x_n - x_m| < \varepsilon$. To make things easier we will
assume that $H > \varepsilon$ but note that this situation can be easily
dealt with . Anyways, suppose such an $H$ exists. Then let $m = H^2$ and
$n = 4H^2$. So we have
\[|x_n-x_m| = |\sqrt{4H^2}-\sqrt{H^2}| = 2H - H = H > \varepsilon.\]
This contradiction shows that the sequence cannot be Cauchy.
\hfill $\blacksquare$

\hrulefill

\textbf{Exercise 3.5 - 6}: Let $p$ be a given natural number. Give an
example of a sequence $(x_n)$ that is not a Cauchy sequence, but that
satisfies $\lim |x_{n+p} - x_n| = 0$.

\underline{\textit{Solution}}: Let $x_n = n$ modulus $p$. Here, any two
terms $p$ apart in index are 0 apart in value, but otherwise they clearly
do not get arbitrarily close. Let $\varepsilon := 1/2$ then for any
$H(\varepsilon)$, and any $n > H(\varepsilon)$,
$|x_{n+1}-x_n| = 1 > \varepsilon$. \hfill $\blacksquare$

\hrulefill

\textbf{Exercise 3.5 - 11}: If $y_1<y_2$ are arbitrary real numbers and
$y_n := \frac{1}{3}y_{n-1} + \frac{2}{3}y_{n-2}$ for $n>2$, show that
$(y_n)$ is convergent. What is its limit?

\underline{\textit{Solution}}: This is a contractive sequence:
\[|y_{n+2}-y_{n+1}|
  = \left| \left( \frac{1}{3}y_{n+1}+\frac{2}{3}y_n \right)
  - y_{n+1} \right|
  = \frac{2}{3} |y_{n+1}-y_n|
\]

Contractive sequences converge. This one converges to $(2/5)y_1+(3/5)y_2$.
This can be verified with monotone-ness and sup and inf on the odd and even
terms of the sequence.

\hrulefill

\textbf{Exercise 3.6 - 5}: Is the sequence $a_n := n \sin n$ properly
divergent?

\underline{\textit{Solution}}: No, $(a_n)$ is not properly divergent.
Suppose that $(a_n)$ trends towards $+\infty$. Then for any
$\alpha \in \mathbb{R}$ there exists a $K(\alpha)$ so that for any
$n > K(\alpha)$, $a_n > \alpha$. Let $\alpha > 0$. Then for at least
one of $\{K+2, K+4, K+6\}$, we know $a_n = n \sin n$ is negative and
therefore less than $\alpha$. A similar argument shows $(a_n)$ does not tend
towards $-\infty$. \hfill $\blacksquare$

\hrulefill

\textbf{Exercise 3.6 - 8}:
Investigate the convergence or divergence of the following sequences:
\begin{enumerate}[(a)]
  \item $(\sqrt{n^2+2})$: This sequence tends towards $+\infty$. Let
    $K(\alpha)$ be the smallest integer greater than $\alpha$. Then for
    any $n \geq K(\alpha)$, $\sqrt{n^2+2} > \sqrt{n^2} = n > \alpha$.
  \item $(\sqrt{n}/(n^2+1))$: Clearly convergent towards 0.
    $0 < \sqrt{n}/(n^2+1) < n/(n^2+1) < n/n^2 = 1/n$.
  \item $(\sqrt{n^2+1}/\sqrt{n})$: Tends towards $+\infty$. Let
    $K(\alpha) > \alpha^2$. Then for any $n > K(\alpha)$,
    $\sqrt{n^2+1}/\sqrt{n} > \sqrt{n^2}/\sqrt{n}
    = \sqrt{n} > \sqrt{\alpha^2} = \alpha$.
  \item $(\sin \sqrt{n})$: This sequence is neither convergent nor
    properly divergent. It is not convergent because if $\varepsilon := 1$
    then for any $n$ there exists a $k$ such that
    $|\sin \sqrt n - \sin \sqrt{n+k}| > 1$. It is not properly divergent
    because it never exceeds +1 or -1.
\end{enumerate}

\hrulefill

\textbf{Exercise 3.7 - 3}: Use partial fraction to show the following
summations.

\begin{enumerate}[(a)]
  \item $\sum\limits_{n=0}^\infty \frac{1}{(n+1)(n+2)} = 1$:
    \begin{flalign*}
      &&s_k :&= \sum_{n=0}^k \frac{1}{(n+1)(n+2)} &\\
      && &= \sum_{n=0}^k \frac{1}{n+1} - \frac{1}{n+2} &\\
      && &= \left( \sum_{n=0}^k \frac{1}{n+1} \right)
      - \left( \sum_{n=0}^k \frac{1}{n+2} \right) &\\
      && &= \left( \sum_{n=0}^k \frac{1}{n+1} \right)
      - \left( \sum_{n=1}^{k+1} \frac{1}{n+1} \right) &\\
      && &= \frac{1}{1} - \frac{1}{k+2} &
    \end{flalign*}
    Clearly $s_k$ converges to 1.
  \item $\sum\limits_{n=0}^\infty \frac{1}{(\alpha+n)(\alpha+n+1)}
    = \frac{1}{\alpha} > 0$, if $\alpha > 0$.
    \begin{flalign*}
      &&s_k :&= \sum_{n=0}^k \frac{1}{(\alpha+n)(\alpha+n+1)} &\\
      && &= \sum_{n=0}^k \frac{1}{\alpha+n} - \frac{1}{\alpha+n+1} &\\
      && &= \left( \sum_{n=0}^k \frac{1}{\alpha+n} \right)
      - \left( \sum_{n=0}^k \frac{1}{\alpha+n+1} \right) &\\
      && &= \left( \sum_{n=0}^k \frac{1}{\alpha+n} \right)
      - \left( \sum_{n=1}^{k+1} \frac{1}{\alpha+n} \right) &\\
      && &= \frac{1}{\alpha} - \frac{1}{\alpha+k+1} &
    \end{flalign*}
    Clearly $s_k$ converges to $1/\alpha$.
  \item $\sum\limits_{n=1}^\infty \frac{1}{n(n+1)(n+2)} = \frac{1}{4}$.
    \begin{flalign*}
      && s_k :&= \sum_1^k \frac{1}{n(n+1)(n+2)} &\\
      && &= \sum_1^k \frac{1}{2n} - \frac{1}{n+1} + \frac{1}{2(n+2)} &\\
      && &= \frac{1}{2} \left( \sum_1^k \frac{1}{n} \right) - \left( \sum_1^k \frac{1}{n+1} \right) + \frac{1}{2} \left( \sum_1^k \frac{1}{n+2} \right) &\\
      && &= \frac{1}{2} \left( \sum_1^k \frac{1}{n} \right) - \left( \sum_2^{k+1} \frac{1}{n} \right) + \frac{1}{2} \left( \sum_3^{k+2} \frac{1}{n} \right) &\\
      && &= \left(\frac{1}{2}\right) + \left(\frac{1}{4} - \frac{1}{2}\right) + \left(\frac{-1}{k+1} + \frac{1}{2(k+1)}\right) + \left(\frac{1}{2(k+2)}\right) &\\
      && &= \frac{1}{4} - \frac{1}{2(k+1)} + \frac{1}{2(k+2)} &
    \end{flalign*}
    Clearly $s_k$ converges to $1/4$.
\end{enumerate}

\hrulefill

\textbf{Exercise 3.7 - 13}: If $\sum a_n$ with $a_n>0$ is convergent, then
is $\sum \sqrt{a_n a_{n+1}}$ always convergent? Either prove it or give a
counterexample.

\underline{\textit{Solution}}: Proving it by contrapositive. Suppose that
$\sum \sqrt{a_n a_{n+1}}$ does not converge. Then neither does
$\sum a_n a_{n+1}$, and by the Cauchy-Schwarz inequality, neither does
$\left(\sum a_n \right) \left( \sum a_{n+1} \right)$. But that last
expression must be finite because it is the product of two finite numbers!
Hence we have a contradiction and $\sum \sqrt{a_n a_{n+1}}$ must converge.
\hfill $\blacksquare$

\hrulefill

\textbf{Exercise 3.7 - 15}: Let $(a(n))$ be a decreasing sequence of
strictly positive numbers and let $s(k) := \sum_{n=1}^k a(n)$. First show
that \[\frac 1 2 (a(1)+2a(2)+\dots+2^na(2^n)) \leq s(2^n)
\leq (a(1)+2a(2)+\dots+2^{n-1}a(2^{n-1})) + a(2^n),\] then use this result
to show that $\sum_{n=1}^\infty a(n)$ converges if and only if
$\sum_{n=1}^\infty 2^n a(2^n)$ converges.

\underline{\textit{Solution}}: TODO

\hrulefill

\textbf{Exercise 3.7-18}: Show that if $c>1$, then the following series are
convergent: 
\begin{enumerate}[(a)]
  \item $\sum \frac{1}{n (\ln n)^c}$: \textit{I looked at the hint in the
    back of the book.} The Cauchy Condensation Test states that $\sum x_n$
    converges if and only if $\sum 2^n x_{2^n}$ converges.
    \[\sum 2^n x_{2^n}
      = \sum \frac{2^n}{2^n (\ln 2^n)^c}
      = \sum \frac{1}{n^c} \frac{1}{(\ln 2)^c}
      = \frac{1}{(\ln 2)^c} \sum \frac{1}{n^c}.
    \]
    The last expression clearly only converges for $c > 1$.
  \item $\sum \frac{1}{n (\ln n) (\ln \ln n)^c}$: We will apply the cauchy
    condensation test twice.
    \begin{flalign*}
      \text{Once:} && y_n := 2^n x_{2^n} &= \sum \frac{2^n}{2^n (\ln 2^n) (\ln \ln 2^n)^c} &\\
      && &= \sum \frac{1}{n \times \ln 2 \times (\ln (n \times \ln 2))^c} &\\
      && &= \sum \frac{1}{n \times \ln 2 \times (\ln n + \ln \ln 2)^c} &\\
      \text{Twice:} && z_n := 2^n y_{2^n} &= \sum \frac{2^n}{2^n \times \ln 2 \times (\ln 2^n + \ln \ln 2)^c} &\\
      && &= \sum \frac{1}{\ln 2 \times (n \times \ln 2 + \ln \ln 2)^c} &\\
      && &= \frac{1}{\ln 2} \sum \frac{1}{(n \times \ln 2 + \ln \ln 2)^c} &\\
      && &< \frac{1}{\ln 2} \sum \frac{1}{(n \times \ln 2)^c} &\\
      && &= \frac{1}{(\ln 2)^{c+1}} \sum \frac{1}{n^c} &
    \end{flalign*}
    Again, this converges for any $c > 1$.
\end{enumerate}

\hrulefill

\textbf{Exercise 4.1 - 7}: Show that $\lim\limits_{x \rightarrow c} x^3=c^3$
for any $c \in \mathbb{R}$.

\underline{\textit{Solution}}: Let $\varepsilon > 0$. Choose
$\delta := \min \{1, \varepsilon/(2c^2+4|c|+1)\}$. Then
\[|x^3-c^3| = |x^2+cx+c| \times |x-c| < ((|c|+1)^2+c^2+|c|) |x-c|
  = (2c^2+4|c|+1) |x-c| < (2c^2+4|c|+1) \frac{\varepsilon}{2c^2+4|c|+1}
  = \varepsilon.
\]
And since we have a way of choosing $\delta$ for any arbitrary assigned
$\varepsilon$, we know the limit is correct. \hfill $\blacksquare$

\hrulefill

\textbf{Exercise 4.1 - 11}: Use the definition of limit to prove the
following:

\begin{enumerate}[(a)]
  \item $\lim\limits_{x \rightarrow 3} \frac{2x+3}{4x-9}=3$: Let
    $\varepsilon>0$, define $f(x):=(2x+3)/(4x-9)$, choose
    $\delta := \min \{1, 17\varepsilon/10\}$, and bound
    $2<x<4$. Then
    \[
      |f(x)-3|
      = \left| \frac{2x+3}{4x+9} - 3 \right|
      = \left| \frac{-10x+30}{4x+9} \right|
      = |x - 3| \times \left| \frac{10}{4x+9} \right|
      < |x - 3| \times \frac{10}{17}
      < \frac{17 \varepsilon}{10} \times \frac{10}{4\times2+9}
      < \frac{17 \varepsilon}{10} \times \frac{10}{17}
      = \varepsilon
    \]
    So since $f(x)$ can get arbitrarily close to 3 as $x$ gets close to 3,
    we can be sure this limit is correct.
    \hfill \small $\blacksquare$ \normalsize
  \item $\lim\limits_{x \rightarrow 6} \frac{x^2-3x}{x+3}=2$: Let
    $\varepsilon > 0$, define $f(x) := (x^2-3x)/(x+3)$, choose
    $\delta := \min \{1, \epsilon\}$, and bound $5<x<7$. Then
    \[|f(x)-2|
      = \left| \frac{x^2-3x}{x+3} -2 \right|
      = \left| \frac{x^2-5x-6}{x+3} \right|
      = |x-6| \times \left| \frac{x+1}{x+3} \right|
      < |x-6| \times \frac{7+1}{5+3}
      = |x-6|
      < \varepsilon
    \]
    Again, this shows the limit. \hfill \tiny $\blacksquare$ \normalsize
\end{enumerate}

\end{document}
