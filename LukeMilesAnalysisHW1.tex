\documentstyle{article}
\title{Homework 1}
\author{Luke Miles}
\setlength{\parindent}{0pt}
\begin{document}
\maketitle
Sorry I'm new at \LaTeX \\

\textbf{Exercise 1.1-14}

$f^{-1}(A \cup B) \\
 = \{y : \exists\ x \in (A \cup B) $\ s.t.\ $ f^{-1}(x) = y\} \\
 = \{y : \exists\ x \in A $\ s.t.\ $ f^{-1}(x) = y\ $\ or \ $ \exists\ x \in B\ $\ s.t.\ $ f^{-1}(x) = y\} \\
 = \{y : \exists\ x \in A $\ s.t.\ $ f^{-1}(x) = y\} \cup \{ \exists\ x \in B\ $\ s.t.\ $ f^{-1}(x) = y\} \\
 = f^{-1}(A) \cup f^{-1}(B) $ \\

$f^{-1}(A \cap B) \\
 = \{y : \exists\ x \in (A \cap B) $\ s.t.\ $ f^{-1}(x) = y\} \\
 = \{y : \exists\ x \in A $\ s.t.\ $ f^{-1}(x) = y\ $\ and \ $ \exists\ x \in B\ $\ s.t.\ $ f^{-1}(x) = y\} \\
 = \{y : \exists\ x \in A $\ s.t.\ $ f^{-1}(x) = y\} \cap \{ \exists\ x \in B\ $\ s.t.\ $ f^{-1}(x) = y\} \\
 = f^{-1}(A) \cap f^{-1}(B) $ \\\\

\textbf{Exercise 1.1-16}

In order to show a bijection, we will show that $f$ is (I) one-to-one and (II) onto.

I.

We will show that $f$ is one-to-one with the existence of an inverse in the appropriate interval.

$
y = x/\sqrt{x^2+1}\\
y^2 = x^2/(x^2+1)\\
y^2 (x^2+1) = x^2\\
y^2 x^2+y^2-x^2 = 0\\
x^2 (y^2-1) + y^2 = 0\\
x^2 = -y^2/(y^2-1)\\
x^2 = y^2/(1-y^2)\\
x = y/\sqrt{1-y^2}\\
$

II.

Since any two inputs that produce the same output are equal, $f$ is also onto:

$
f(x_1)=f(x_2)\\
x_1/\sqrt{x_1^2+1}=x_2/\sqrt{x_2^2+1}\\
x_1^2/(x_1^2+1)=x_2^2/(x_2^2+1) \\
x_1^2 (x_2^2+1) = x_2^2 (x_1^2+1) \\
x_1^2 x_2^2 + x_1^2 = x_2^2 x_1^2 + x_2^2 \\
x_1^2 = x_2^2 \\
x_1 = x_2
$\\\\

\textbf{Exercise 1.2-6}

We will use induction.\\

Define $f(x):=x^3+5x$.\\
Clearly, $f(1)=1^3+5*1=1+5=6$ is divisible by 6.\\
Now suppose that $f(i)$ is divisible by 6 for all $i \leq k$.\\
Since\\
$f(k+1)-f(k)\\
=(k+1)^3+5(k+1)-(k^3+5k)\\
=k^3+3k^2+3k+1+5k+5-k^3-5k\\
=3k^2+3k+6\\
=3(k^2+k+2)$\\
is divisible by 6 ($k^2+k+2$ is always even), $f(k+1)$ is also divisible by 6\\
By induction, 6 divides $f(n)$ for all natural $n$.\\\\

\textbf{Exercise 1.2-10}

We will show that $1/(1 \times 3)+1/(3 \times 5) + \ldots + 1/((2n-1)(2n+1)) = n(2n+1)$.

The base case is trivial.

Now suppose the relation is true for $i \leq k$. We will look at $k+1$.\\
$k/(2k+1)+1/((2(k+1)-1)(2(k+1)+1))\\
=k/(2k+1)+1/((2k+1)(2k+3))\\
=k(2k+3)/((2k+1)(2k+3))+1/((2k+1)(2k+3))\\
=(2k^2+3k+1)/((2k+1)(2k+3))\\
=(2k+1)(k+1)/((2k+1)(2k+3))\\
=(k+1)/(2(k+1)+1)$

\textbf{Exercise 1.2-18}

Using induction. Base case is trivial.

Showing that the LHS increases more than the RHS when you add a term:

$
\sqrt{n(n+1)}>\sqrt{n \times n}=n \\
\sqrt{n(n+1)}+1>n+1 \\
\sqrt{n} + 1/\sqrt{n+1} > \sqrt{n+1}
$

\end{document}
